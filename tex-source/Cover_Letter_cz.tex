\documentclass[12pt,a4paper,roman, colorlinks, linkcolor=cyan]{moderncv}

\moderncvstyle{casual} 
\moderncvcolor{blue} 

\setlength{\tabcolsep}{12pt}

\usepackage[utf8]{inputenc} 
\usepackage[czech]{babel}
\usepackage{fontawesome}
\usepackage{tabularx} 
\usepackage{xcolor} 
\usepackage{ragged2e}

\usepackage[scale=0.8]{geometry}

\name{Dominik,}{Bálint}

\title{Resumé title}

\phone[mobile]{+420~000~000~000}
\email{email@email.cz}
\homepage{https://dominik-97.github.io/curriculum-vitae/}
\social[linkedin]{dominik-bálint}

\setlength{\footskip}{66pt}

\begin{document}

\hypersetup{urlcolor= blue}

\recipient{ABB}{\_ADDRESS\_\\\_ADDRESS\_\\\_ADDRESS\_} 
\date{\today} 
\opening{Vážený pane/paní. ,} 
\closing{S přáním příjemného dne,}
\enclosure[Přiloženo]{curriculum vit\ae{}}

\makelettertitle

Úvodní odstavec: Jaký je váš záměr při psaní tohoto dopisu? Co
o kterou pozici se ucházíte a jak jste se o ní dozvěděli? Krátce
představte sebe, svého majora a očekávaný stupeň. Pokud jste
víte o konkrétním otevření, podívejte se na něj. Pokud si nejste vědomi a
konkrétní pozici, uveďte svou oblast zájmu. Tento odstavec může
také se používá k označení osoby, která vám doporučila
kontaktujte organizaci nebo jiné faktory, které vás k psaní přiměly.
Pokud je to možné, sdělte, proč vás zajímá organizace a
cokoli, co víte o jejich produktu nebo službě.

Druhý odstavec: Jaká je vaše kvalifikace? Proč chceš
pracovat pro tuto organizaci? Co byste pro ně rádi dělali? Prodat
sebe a buďte stručný. Povzbuďte zaměstnavateli chuť k jídlu
bude si chtít přečíst váš životopis a naplánovat si pohovor. Popsat
zvýraznění z vašeho pozadí, která by vás nejvíce zajímala
organizace. Zaměřte se na dovednosti, činnosti, úspěchy a
minulé zkušenosti můžete přispět k organizaci a její práci.
Pokud je to možné, prokažte, že něco o
organizace a průmysl / pole. Použijte akční slovesa, která popisují
příslušné dovednosti a odborné znalosti, kterými můžete přispět. Uveďte konkrétní
znalosti, které můžete mít, například počítačové aplikace, cizí
jazyky, laboratorní techniky, možnosti psaní a úpravy. Ty jsi
pokus o vytvoření shody nebo „představy o vhodnosti“ mezi zaměstnavatelem
najímání potřeb a vašich zájmů, zkušeností a dovedností

Třetí odstavec: Jaký je váš akční plán? Chcete navázat
s telefonním hovorem nebo chcete, aby vás kontaktovali? Zavřete svůj
dopis s uvedením, že byste chtěli diskutovat o zaměstnání
příležitosti nebo jiné informace s jednotlivcem a že vy
zavolá, aby na váš dopis navázal. To ukazuje vaše
iniciativu a následnou kontrolu a pomůže vám udržet si určitou kontrolu
svého úsilí.

Další body, které lze uvést v posledním odstavci: • Vyjádřete svůj
ochota poskytnout další informace • Uveďte konkrétní čas
kdy budete kontaktovat telefonicky nebo e-mailem • Sdělte jim, zda a kdy
se chystáte navštívit jejich oblast • Poděkujte osobě, která obdrží vaši
dopis za jejich čas a zájem Co je nejdůležitější, nezapomeňte
adresujte motivační dopis osobě. Pokud nemáte jméno, zavolejte
oddělení nebo lidské zdroje, abyste zjistili, komu je váš dopis
by měl být řešen. Jako poslední možnost adresujte svůj dopis na adresu
personální manažer, náborový manažer nebo náborový zástupce.

\makeletterclosing

\end{document}