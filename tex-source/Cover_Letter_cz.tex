\documentclass[12pt,a4paper,roman, colorlinks, linkcolor=cyan]{moderncv}

\moderncvstyle{casual} 
\moderncvcolor{blue} 

\setlength{\tabcolsep}{12pt}

\usepackage[utf8]{inputenc} 
\usepackage{fontawesome}
\usepackage{tabularx} 
\usepackage{xcolor} 
\usepackage{ragged2e}

\usepackage[scale=0.8]{geometry}

\name{Dominik,}{Bálint}

\title{Resumé title}

\phone[mobile]{+420~000~000~000}
\email{john@doe.org}
\homepage{www.johndoe.com}
\social[linkedin]{john.doe}

\setlength{\footskip}{66pt}

\begin{document}

\hypersetup{urlcolor= blue}

\recipient{ABB}{\_ADDRESS\_\\\_ADDRESS\_\\\_ADDRESS\_} 
\date{\today} 
\opening{Dear Mr. ,} 
\closing{Yours sincerely,}
\enclosure[Attached]{curriculum vit\ae{}}

\makelettertitle

Opening Paragraph: What is your intent in writing this letter? What
position are you applying for and how did you learn about it? Briefly
introduce yourself, your major, and the degree anticipated. If you are
aware of a specific opening, refer to it. If you are not aware of a
specific position, state your area of interest. This  paragraph can
also be used to refer to the individual who  recommended that you
contact the organization, or other factors that prompted you to write.
If possible, convey why you are  interested in the organization and
anything you know about their product or service.

Second Paragraph: What are your qualifications? Why do you want to
work for this organization? What would you enjoy doing for them? Sell
yourself and be brief. Whet the employer’s appetite so that he/she
will want to read your resume and schedule an interview. Describe
highlights from your background that would be of greatest interest to
the organization. Focus on skills, activities, accomplishments, and
past experience you can contribute to the organization and its work.
If possible, demonstrate that you know something about the
organization and industry/field. Use action verbs that describe
relevant skills and expertise you can contribute. Mention specific
knowledge you may have such as computer applications, foreign
languages, lab techniques, writing and editing capabilities. You are
attempting to create a match or “notion of fit” between the employer’s
hiring needs and your  interests, experience, and skills

Third Paragraph: What is your plan of action? Do you want to follow up
with a phone call or do you want them to contact you? Close your
letter by stating that you would like to discuss employment
opportunities or other information with the individual and that you
will call to follow up on your letter. This demonstrates your
initiative and follow-through and will help you maintain some control
of your efforts.

Other points that can be made in the last paragraph: • Express your
willingness to provide additional information • State a specific time
when you will follow up by phone or email • Let them know if and when
you are going to visit their area • Thank the person receiving your
letter for their time and interest Most importantly, remember to
address the cover letter to a person. If you do not have a name, call
the department or human resources to find out to whom your letter
should be addressed. As a last resort, address your letter to the
personnel manager, hiring manager, or recruiting representative.

\makeletterclosing

\end{document}